%------------------------------------------------------------------------------
\section{The instrumental variable quantile regression model} \label{sec:model}
%------------------------------------------------------------------------------

%------------------------------------------------------------------------------
	\subsection{The model}
%------------------------------------------------------------------------------
The general instrumental variables quantile regression model was first
proposed by \cite{Chernozhukov2005}. {\ivqreg} is based on the linear IVQR model
described in \cite{Chernozhukov2006} and \cite{Chernozhukov2008}.

To simplify notation, we use the capital letter to denote the random variable
and a lower-case letter to denote the random variable's actual value.  We use
the bold letter to represent a vector. All the vectors are assumed to be  column
vectors.

The linear IVQR model can be written in the form of ``random
coefficients'' model
\begin{align}
Y &= \Db' \alphab(\UD) + \Xb' \betab(\UD) \quad \text{where  $\UD|\Xb, \Zb \sim
Uniform(0,1)$}  \label{eq:outcome} \\
\Db &= \delta(\Xb, \Zb, V) \quad \text{where $V$ statistically depends on $\UD$} 
\label{eq:dvar} \\
\tau &\rightarrow \Db' \alphab(\tau) + \Xb' \betab(\tau) 
\quad \text{is strictly increasing in $\tau$}
\end{align}
where
\begin{itemize}
\item $Y$ is a scalar outcome variable,
\item $\UD$ is a scalar random variable that charaterizes the heterogeneity of
the outcome and captures all the unobservables in the outcome Equation
\ref{eq:outcome}, 
\item $\Db$ is a vector of endogenous variables that statistically depend on
$\UD$, 
\item $\Xb$ is a vector of exogenous variables that are independent of $\UD$,
\item $\alphab()$ and $\betab()$ are random coefficient vectors that depend on
$\UD$, 
\item the endogenous variables $\Db$ are dermined via Equation \ref{eq:dvar},
\item $\Zb$ is a vector of instrumental variables that are independent of $\UD$
but correlated with $\Db$, 
\item $V$ is a scalar unobserved random variable that impacts $\Db$ and also
correlated with $\UD$, and
\item the observable variables are $\{Y_i, \Xb_i, \Db_i, \Zb_i\}_{i=1}^n$ with a
sample of size $n$.
\end{itemize}

There are two objectives of the analysis.
\begin{enumerate}

\item Estimate the conditional quantile function of the latent potential outcome
$\Yd$ when fixing $\Db = \db$ and conditional on $\Xb$. More precisely, the
potential outcome $\Yd = \db'\alpha(\Ud) + \Xb'\betab(\Ud)$ with $\Ud$ as a
scalar random variable from uniform distribution conditional on $\Xb$ (note that
$\db$ are treated as constant here). The conditional quantile function of $\Yd$
can be written as
\begin{align}
S_{\Yd}(\Yd | \tau, \Xb, \db) = \db'\alphab(\tau) + \Xb'\betab(\tau)
\end{align}
Notice that $S_{\Yd}()$ is generally different from the conditional quantile
function for the observed outcome $Y$ because $\Db$ are endogenous. $S_{\Yd}()$
is also referred to as structural quantile equation (SQE).


\item Estimate the quantile treatment effects of $\Db$. Suppose $\Db$ is a
binary variable that can only take $0$ and $1$, The quantile treatment
effects (QTE) are defined as 
\begin{align}
S_{Y_1}(Y_1 | \tau, \Xb, 1) - S_{Y_0}(Y_0 | \tau, \Xb, 0) = \alphab(\tau)
\end{align}
Or, if $\Db$ is continuous, QTE is defined as
\begin{align}
\frac{\partial S_{Y_{\db}}()}{\partial \db} = \alphab(\tau)
\end{align}

The linear-in-parameter assumption greatly simplifies the estimation of QTE, which
is $\alphab(\tau)$ in both discrete and continuous case.  
\end{enumerate}

Suppose we estimate the function $S_{\Yd}()$ at different values of $\tau$
spanned between $0$ and $1$. In that case, we can have a fuller picture of the
conditional distribution of $\Yd$ than just estimating the mean of the
distribution. In addition, the estimates for $\alphab(\tau)$ at different values
of $\tau$ reveal how the treatment effects vary at different conditional
quantile indexes.



%------------------------------------------------------------------------------
	\subsection{Main assumptions and moment condition}
	\label{sec:cond}
%------------------------------------------------------------------------------
The primary condition in the IVQR model (\cite{Chernozhukov2005}) is

\begin{assumption}
  \label{assum:ivqr}

Consider a probability space $(\Omega, F, P)$ and the set of potential outcome
variables $(\Yd, d \in \mathbbm{D})$, endogenous variables $\Db$, exogenous
covariates $\Xb$, and instruments $\Zb$. The following conditions hold:

\begin{description}

\item[A1] (Potential outcomes) Conditional on $\Xb=\xb$, for each $\db$, $\Yd =
q(\db, \xb, \Ud)$, where $q(\db, \xb, \tau)$ is increasing in $\tau$ and $\Ud
\sim U(0, 1)$.

\item[A2] (Independence) Conditional on $\Xb=\xb$, $\Ud$ are independent of
$\Zb$.

\item[A3] (Selection) $\Db = \delta(\Zb, \Xb, V)$ for some unknown function
$\delta(\cdot)$ and random vector $V$.

\item[A4] (Rank similarity) Conditional on $(\Xb, \Zb, V)$, $\{\Ud\}$ are
identically distributed.

\item[A5] (Observables) Observed variables consist of $Y = q(\Db, \Xb, \UD)$,
$\Db$, $\Xb$, and $\Zb$.

\end{description}
\end{assumption}

In the linear IVQR model, $q(\Db, \Xb, \UD) = \Db'\alphab(\UD) +
\Xb'\betab(\UD)$ as specified in Equation \ref{eq:outcome}.

The following is the main implication of Assumption A1-A5.

\begin{theorem}[Main implications of IVQR model]\label{th:moment}
Suppose conditions A1-A5 hold. Then for all $\tau \in (0, 1)$, a.s.
\begin{align}
	P(Y \leq q(\Db, \Xb, \UD) | \Xb, \Zb) = \tau
	\label{eq:moment1}
\end{align}
and $\UD \sim U(0,1)$ conditional on $\Xb$ and $\Zb$.
\end{theorem}

For the proof of Theorem \ref{th:moment} and a discussion on the rank
similarity, see Section \ref{sec:proof_moment} in Appendix.  Theorem
\ref{th:moment} also implies the following unconditional moment condition that
can be used to estimate the IVQR model. Namely, Equation \ref{eq:moment1}
together with the linear-in-parameter assumptions imply

\begin{align}
\E \left[ \left( \tau - \I (Y  - \Db'\alphab - \Xb'\betab \leq 0)
	\right) \Psib(\Xb, \Zb) \right]= 0 \label{eq:moment2}
\end{align}
where $\I()$ is the indicator function, $\Psib(\Xb, \Zb)$ is some transformation
of $\Xb$ and $\Zb$. In practice, $\Psib(\Xb, \Zb) = (\Phib(\Xb, \Zb)',
  \xb')'$, and $\Phib(\Xb, \Zb)$ is the projection of
$\Db$ into the linear space spanned by $\Xb$ and $\Zb$.

The empirical version of Equation \ref{eq:moment2} is
\begin{align}
\frac{1}{n} \sum_{i=1}^n \left[ 
	\left( 
	\tau - \I (Y_i  - \Db_i'\alphab - \Xb_i'\betab \leq 0) 
	\right) 
	\Psib(\Xb_i, \Zb_i) 
	\right]= 0 
\label{eq:moment3}
\end{align}

Equation \ref{eq:moment2} and \ref{eq:moment3} are the foundation for different
estimation strategies, that will be discussed in Sections \ref{sec:rev},
\ref{sec:iqr_method}, \ref{sec:see_method}, and \ref{sec:dec_method}.
