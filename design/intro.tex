%------------------------------------------------------------------------------
\section{Introduction}
%------------------------------------------------------------------------------

In empirical applications, we are usually interested in the causal effects of
some covariate on the outcome variable. The traditional linear regression model
is an excellent way to model how the covariate affects the outcome's conditional
mean. 
However, sometimes we would like to study features of the outcome distribution
different than the mean to have a full picture of the causal effects of
covariates.
For example, a policy maker may be more interested in how a summer
job training program affects the income's lower quantile instead of just
its mean. 

Quantile regression in \cite{Koenker1978} can help us grasp a better picture
than the regular linear regression by estimating the causal effects on different
quantiles of the outcome's conditional distribution. 
For a discussion on quantile regression, see {\tt qreg}.

In practice, some covariates of interests are often endogenous due to various
reasons such as self-selection, omission of some relevant variable, and
measurement error. For example, suppose we are interested in how the
participation of the 401(k) program affects the net wealth. However,
participation in the 401(k) program is endogenous because the people who do and
do not participate may have different saving preferences, which will affect the
net wealth growth. 

Endogenous covariates make quantile regression estimates inconsistent, as
is the case for linear regression model. Analogous to the instrumental
variable least square estimator, there are instrumental variable quantile
regression estimator to compute the different quantiles of casual effects
consistently. For a discussion of instrumental variables estimation, see
{\tt ivregress}.

{\ivqreg} fits a quantile regression model with endogeneity using
two estimators: the inverse quantile regression estimator proposed in
\cite{Chernozhukov2006} and the smoothed estimating equation estimator outlined
in \cite{Kaplan2017}.  Intuitively, {\ivqreg} can be thought of as the {\tt
ivregress} version of the {\tt qreg}, although the underlying estimators are not
as straightforward as the two-stage least-squares estimator. 

This paper describes the design of the {\ivqreg} and a suite of post-estimation
tools to estimate, visualize, make the inference, and diagnose the instrumental
variable quantile regression model. In particular, the Stata commands in the
IVQR toolbox can be grouped into the following categories.

\begin{description}
  \item{\bf Estimation}
    \begin{itemize}
      \item {\tt ivqregress iqr} estimates the IVQR model by the inverse
	quantile regression (IQR) estimator proposed in \cite{Chernozhukov2006}.
	\item {\tt ivqregress smooth} estimates the IVQR model by the smoothed
	  estimation equation (SEE) estimator proposed in \cite{Kaplan2017}.
    \end{itemize}

  \item{\bf Visualization}
    \begin{itemize}
      \item {\tt estat coefplot} visualizes how the treatment effects vary at
	different quantiles of the outcome.

	\item {\tt marginsplot} plots the potential outcome's conditional
	  quantile function.
    \end{itemize}

  \item{\bf Inference} 
    \begin{itemize}
      \item {\tt estat endogeffects} makes inference of particular interest in
	the context of IVQR model. In particular, {\tt estat endogeffects} test
	four hypotheses: 1. the quantile treatment effects are zero; 2. the
	quantile treatment effects are  constant across different quantiles; 3.
	the quantile treatment effects are unambiguously beneficial; 4. the
	treatment is exogenous.

     \item {\tt estat dualci} provides confidence interval robust to weak
       instruments for the endogenous treatment effects. It is allowed only
       after {\tt ivqregress iqr}.

      \item {\tt test} and {\tt testnl} infer classical linear and
	nonlinear hypotheses after estimation.

      \item {\tt margins} helps to compute the potential outcome's conditional
	quantile function.

    \end{itemize}

   \item{\bf Diagnosis}

     \begin{itemize}
       \item {\tt estat waldplot} helps diagnosis the convergence of the
	 inverse quantile regression estimator ({\tt ivqregress iqr}). In
	 particular, {\tt estat waldplot} visualizes the optimization process
	 during the computation in {\tt ivqregress iqr} and shows if the
	 searching domain contains the true value of the parameter with a
	 predefined probability level.
     \end{itemize}

\end{description}

We organize this paper as follows. Section \ref{sec:model} describes the
instrumental variable quantile regression model and the primary moment condition
for estimation. Section \ref{sec:rev} reviews the existing estimators, discusses
their strength and weakness and identify two estimators implemented in
{\ivqreg}.  In particular, {\ivqreg} implements the inverse quantile regression
estimator and the smoothed estimating equation estimator, discussed in
Sections \ref{sec:iqr_method} and \ref{sec:see_method}, respectively.  Section
\ref{sec:infer} outlines the general inference approach of particular
interest in the IV quantile regression model. Section \ref{sec:ivqreg_syntax}
describes the syntax of {\ivqreg}. Section \ref{sec:post_syntax} describes the
post-estimation of {\ivqreg}. Finally, section \ref{sec:example} illustrates the
use of {\ivqreg} and its post-estimation through some examples. 


