%------------------------------------------------------------------------------
\subsection{Proof for Theorem \ref{th:moment}}	 \label{sec:proof_moment}
%------------------------------------------------------------------------------
\begin{proof}

\begin{align*}
P(Y \leq q(\Db, \Xb, \tau) |\Xb=\xb, \Zb=\zb)
	& \stackrel{(1)}{=} P( q(\Db, \Xb, \UD) \leq q(\Db, \Xb, \tau) |
	\Xb=\xb, \Zb=\zb) \\
	& \stackrel{(2)}{=} P( \UD \leq \tau | \Xb=\xb, \Zb=\zb) \\
	& \stackrel{(3)}{=} \int P(\UD \leq \tau | \Xb=\xb, \Zb=\zb, V=v)
		dP(V=v|\Xb=\xb, \Zb=\zb) \\
	& \stackrel{(4)}{=} \int P(U_{\delta(\xb, \zb, v)} \leq \tau
		| \Xb=\xb, \Zb=\zb, V=v) dP(V=v|\Xb=\xb, \Zb=\zb)  \\
	& \stackrel{(5)}{=} \int P(\Ud \leq \tau
		| \Xb=\xb, \Zb=\zb, V=v) dP(V=v|\Xb=\xb, \Zb=\zb)  \\
	& \stackrel{(6)}{=} P(U_0 \leq \tau | \Xb=\xb, \Zb=\zb) \\
	& \stackrel{(7)}{=} \tau
\end{align*}
Equality (1) by the definition of $Y$ in A5. Equality (2) follows the decreasing
feature of $q()$ in A1. Equality (3) is by the definition of conditional
probability. Equality (4) is by the definition of $\Db$ in A3. Equality (5) holds
because, conditional on $(\Xb, \Zb, v)$, $\Db$ is a constant, and the
distribution of $\Ud$ is identical. Here, we assume $\Db = 0$ in this case.
$\Db$ can be any fixed value in $\mathbbm{D}$. Equality (6) is by definition.
Finally, equality (7) holds because $\Ud \sim U(0,1)$ in A1 and $\Ud$ is
independent of $\Xb$ and $\Zb$ in A2.

\begin{align*}
P(\UD \leq \tau | \Xb=\xb, \Zb=\zb) &= P(U_{\delta(\Xb, \Zb, V)} \leq \tau |\Xb
= \xb, \Zb=\zb) \\
	&= \int P(U_{\delta(\Xb, \Zb, v)}\leq \tau |\Xb=\xb, \Zb=\zb, V=v)
		d P(V=v|\Xb, \Zb)  \\
	&= \int P(U_0 \leq \tau | \Xb=\xb, \Zb=\zb, V=v) d P(V=v | \Xb, \Zb) \\
	&= P(U_0 \leq \tau | \Xb=\xb, \Zb=\zb)
\end{align*}
So conditional on $\Xb$ and $\Zb$, $\UD$ has the same distribution as $\Ud$ for
a fixed value of $\db$.

\end{proof}

From the proof for Theorem \ref{th:moment}, we see that rank similarity
assumption in A4 is the fundamental assumption to convert the conditional
distribution of $\UD$ to $\Ud$ with fixed value $\db$. Here, we discuss the
nuisance underlying this assumption.

First, rank invariance is a particular case of rank similarity. Namely, the
rank invariance assumes $\Ud$ are the same across different values in
$\mathbbm{D}$. While convenient and supported by some applications, the rank
invariance assumption may be too strong in practice.

Second, rank similarity means that given an assignment of treatment, the rank
$\Ud$ is identically distributed. For example, among the individuals who have
$\Xb=\xb$, $\Zb=\zb$, and $D=1$, the distribution of $U_0$ and $U_1$ are the
same. In this formulation, we implicitly assume that one selects the treatment
without knowing the potential outcomes. However, the individuals may know the
distribution of the potential outcomes given $(\Xb, \Zb, v)$ but not the exact
value of $U_0$ for each observation.
