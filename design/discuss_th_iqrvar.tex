%------------------------------------------------------------------------------
\subsection{Discussion on Theorem \ref{thm:iqr_var}} \label{sec:discuss_iqrvar}
%------------------------------------------------------------------------------

For a rigorous proof for Theorem \ref{thm:iqr_var}, see \cite{Chernozhukov2006}.
Here I provide a intuitive interpretation. IQR estimator approximately solves
the GMM moment condition in Equation \ref{eq:moment2}, and thus the
variance of IQR estimator can be understood via the GMM approach. We take the
first-order taylor expansion of the sample analog of moment
\ref{eq:moment3} at the true value $\thetab(\cdot)$.  The term,
$\frac{1}{\sqrt{n}} \sum_{i=1}^n l_i(\cdot, \thetab(\cdot))\Psib_i(\cdot)$, is
the moment condition at the true value. The term, $\Jb(\cdot)$ is the gradient of
the moment with respect to $\alphab$ and $\betab$. The term, $\Sb(\tau, \tau')$ is
the variance for the moment
$$
\E \left[ l(\cdot)\Psib(\cdot) l(\cdot)\Psib(\cdot)' \right]
$$
This variance has a block-by-block structure such that the block for the $(\tau,
\tau')$ is $\Sb(\tau, \tau')$.
\begin{align*}
\E &\left[ l(\tau)\Psib(\tau) l(\tau')\Psib(\tau')' \right] \\
 &=  \E \left[ l(\tau) l(\tau')\Psib(\tau)\Psib(\tau')' \right] \\
& = \E\left\{
\E \left[ l(\tau) l(\tau')\Psib(\tau)\Psib(\tau')' |\Xb, \Zb\right]
\right \} \\
& = \E \left \{
\E \left[
l(\tau) l(\tau') | \Xb, \Zb
\right] \Psib(\tau) \Psib(\tau')'
\right \} \\
& = \E \left \{
\E \left[
(\tau - \I (\epsilon(\tau) \leq 0)) (\tau' - \I (\epsilon(\tau') \leq 0)
| \Xb, \Zb
\right] \Psib(\tau) \Psib(\tau')'
\right \} \\
& = \E \left \{
\E \left[
(\tau\tau' - \tau\I (\epsilon(\tau')\leq 0)
- \I(\epsilon(\tau)\leq 0)\tau' + \I (\epsilon(\tau)\leq 0)
\I (\epsilon(\tau') \leq 0)
| \Xb, \Zb
\right] \Psib(\tau) \Psib(\tau')'
\right \}
\intertext{notice that $\E(\I(\epsilon(\tau) \leq 0) = \tau$ by the main
implications in IVQR model in Equation \ref{eq:moment1}. So}
& = \E \left \{
\E \left[
(\tau\tau' - \tau \tau'
- \tau\tau' + \I (\epsilon(\tau)\leq 0) \I (\epsilon(\tau' \leq 0))
| \Xb, \Zb
\right] \Psib(\tau) \Psib(\tau')'
\right \}
\intertext{smaller value of $\tau$ also means greater residuals $\epsilon$, so
$\I (\epsilon(\tau)\leq 0) \I (\epsilon(\tau' \leq 0))$ is simplified to
$\min(\tau, \tau')$}
& = \E \left \{
\E \left[ (\min(\tau, \tau') - \tau\tau')  | \Xb, \Zb \right] 
\Psib(\tau) \Psib(\tau')'
\right \} \\
& =
(\min(\tau, \tau') - \tau\tau') \E \left [ \Psib(\tau) \Psib(\tau')' \right]
\end{align*}

For the gradient, $\Jb(\cdot)$,
\begin{align*}
\Jb(\tau) &= \frac{\partial \E(l(\tau) \Psib(\tau))}{\partial \thetab'} \\
&= \E \left[ \E \left( \frac{\partial
l(\tau) \Psib(\tau)
}{\partial \thetab'} |\Xb, \Db, \Zb \right) \right] \\
&= \E \left[ \E \left( \frac{\partial
l(\tau)
}{\partial \thetab'} |\Xb, \Db, \Zb \right) \Psib(\tau) \right] \\
&= \E \left[  \frac{\partial \E ( l(\tau) |\Xb, \Db, \Zb )
}{\partial \thetab'}  \Psib(\tau) \right] \\
&= \E \left[ \left( \frac{\partial F_{\epsilon}(0 |\Xb, \Db, \Zb)
}{\partial \thetab'}  \right) \Psib(\tau) \right] \\
&= \E \left[f_{\epsilon}(0 |\Xb, \Db, \Zb) \Psib(\tau) [\Db', \Xb'] \right] \\
\end{align*}
