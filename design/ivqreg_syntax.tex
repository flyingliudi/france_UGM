
%------------------------------------------------------------------------------
\section{Syntax of {\ivqreg}} \label{sec:ivqreg_syntax}
%------------------------------------------------------------------------------
{\ivqreg} fits a linear IV quantile regression model using the
inverse quantile regression  estimator or the smoothed estimating equation
estimator.

\subsection{Syntax}
\vskip 0.5cm
\noindent
inverse quantile regression estimator:
\vskip 0.5cm

\begin{tabular}{ll}
  \ivqregiqr & {\it depvar } [{\it $varlist_1$}]
	({\it $varname_2 = varlist_{iv}$}) [{\tt if}] [{\tt in}] 
	[ , {\it options} \hskip 0.2cm
	{\it IQR\_options} ]
\end{tabular}

\vskip 0.5cm

\noindent
smoothed estimating equation estimator:
\vskip 0.5cm

\begin{tabular}{ll}
  \ivqregsee & {\it depvar } [{\it $varlist_1$}] 
	({\it $varlist_2 = varlist_{iv}$}) [{\tt if}] [{\tt in}] 
	[ , {\it options} \hskip 0.2cm {\it SMOOTH\_options} ]
\end{tabular}

\vskip 0.5cm
\noindent
where
\begin{itemize}
\item {\it $varlist_1$} specifies the exogenous variables.
\item {\it $varname_2$} or {\it $varlist_2$}  specify the endogenous variables.
\item {\it $varlist_{iv}$} specifies the instrumental variables.
\item Factor variables are allowed for {\it $varlist_1$}, {\it $varlist_2$}, or
  {\it $varname_2$}.
\end{itemize}

\noindent
The options are
\vskip 0.3cm

\begin{tabular}{ll}
\hline
{\it options} & Description\\
\hline
{\bf Model} \\
\quad {\tt \underline{q}uantile(numlist)} & estimate quantiles specified in
{\tt numlist}; \\
& default is {\tt quantile(0.5)} \\
\\
{\bf SE/Robust} \\
% \quad {\tt vce(\underline{r}obust [,{\it robustopts}])} & robust standard
% errors, the default \\
\quad {\tt vce({\it vcespec})} & technique used to estimate standard errors;  \\
& the default is {\tt vce(robust)} \\
{\bf Optimization} \\
\quad {\tt [no]log} & suppress or display the iteration log \\
\quad {\tt verbose} & display a verbose iteration log \\
\\
{\bf Reporting} \\
\quad {\tt \underline{l}evel(\#)} & set confidence level; default is {\tt
level(95)} \\
\quad {\it display\_options} & control display formats \\
\hline
\end{tabular}

\vskip 0.4 cm
Notes:
\begin{itemize}

  \item The prefixes {\tt collect}, {\tt by}, {\tt statsby}, {\tt rolling}, {\tt
    bootstrap} and {\tt xi} are allowed.

\item The {\it vcespec} can be only one of the following
  	
\qquad  {\tt \underline{r}obust} [, {\it robust\_options}]

\qquad 	or 

\qquad	{\tt \underline{boot}strap [, {\it bootstrap\_options}]}

where {\it boostrap\_options} are options for {\tt bootstrap}. See {\tt help
vce\_option}.

The default is {\tt robust}.

\item The {\it robust\_options} is 

\begin{tabular}{ll} 
\hline
{\tt \underline{k}ernel({\it kernel})} & use a nonparametric kernel
density estimator; \\
& default is {\tt epanechnikov} \\
{\tt \underline{bw}idth(\#|{\it bwrule})} & bandwidth used by the kernel
density estimator;\\ & default is Silverman's rule of thumb\\
\hline
\end{tabular}

\item The {\it kernel} is

\begin{tabular}{ll}
\hline
{\it kernel} & Description \\
\hline
{\tt \underline{ep}anechnikov}& Epanechnikov kernel function; the default \\
{\tt epan2       	     }   & alternative Epanechnikov kernel function \\
{\tt \underline{bi}weight    }& biweight kernel function \\
{\tt \underline{cos}ine      }& cosine trace kernel function \\
{\tt \underline{gau}ssian    }& Gaussian kernel function \\
{\tt \underline{par}zen      }& Parzen kernel function \\
{\tt \underline{rec}tangle   }& rectangle kernel function \\
{\tt \underline{tri}angle    }& triangle kernel function \\
\hline
\end{tabular}

\item The {\it bwrule} is

\begin{tabular}{ll}
\hline
{\it kernel} & Description \\
\hline
{\tt \underline{si}lverman} & Silverman rule of thumb; the default \\
{\tt \underline{hs}heather} & Hall–Sheather’s bandwidth \\
{\tt \underline{bo}finger} & Bofinger’s bandwidth \\
\hline
\end{tabular}

\end{itemize}

\vskip 0.5cm
\noindent
The {\it IQR\_options} are
\vskip 0.5cm

\begin{tabular}{ll}
\hline
{\it IQR\_options} & Description\\
\hline
{\tt \underline{ng}rid(\#$_g$)} & use \#$_g$ grid points; default is {\tt
grid(30)} \\
{\tt \underline{bou}nd(\#$_{min}$ \#$_{max}$ [, at(\#$_{q}$)])} &
specify the lower and the upper bound for the grid \\
& in the \#$_{q}$-th quantile estimation; may be repeated \\
% {\tt \underline{noadapt}ive} & do not adaptively search the grid points
% using the dual CI;
% \\ &default is conducting an adaptive grid search\\
\hline
\end{tabular}

\vskip 0.5 cm
Notes:
\begin{itemize}
 
  \item The option {\tt bound()} can be specified only once for each quantile
    specified in option {\tt quantile()}. If {\tt at()} is not specified, the
    bound is applied to all the quantiles specified in option {\tt
    quantile()}.

\end{itemize}

\vskip 0.5cm
\noindent
The {\it SMOOTH\_options} are
\vskip 0.5cm

\begin{tabular}{ll}
\hline
{\it SMOOTH\_options} & Description\\
\hline
{\tt \underline{bw}idth(\#$_b$ [, at(\#$_{q}$)])} &
specify bandwidth \#$_b$ to smooth the estimating equation \\
& for the \#$_q$-th quantile estimation; the default is \\
&the theoretical optimal bandwidth; may be repeated
\\
{\tt nosearchbwidth}&	
do not search for feasible bandwidth if the initial bandwidth \\
& is not feasible; default is searching for feasible bandwidth \\
{\tt \underline{iter}ate(\#)} & perform maximum of \# iterations when solving
estimating equation; \\ &default is iterate(100)\\
{\tt \underline{tol}erance(\#)} & 
tolerance for the coefficient vector; default is {\tt tolerance(1E-9)} \\
{\tt \underline{ztol}erance(\#)} & tolerance used to determine 
whether the proposed solution \\ 
& to a zero-finding problem is sufficiently close to zero; \\
& default is {\tt ztolerance(1E-9)} \\
\hline
\end{tabular}

\vskip 0.5 cm
Notes:
\begin{itemize}
 
  \item The option {\tt bwidth()} can be specified only once for each
    quantile specified in option {\tt quantile()}.  If {\tt at()} is not
    specified, the bandwidth is applied to all the quantiles specified in option
    {\tt quantile()}.

\end{itemize}

%------------------------------------------------------------------------------
\subsection{Quick start}
%------------------------------------------------------------------------------

%%%%%%%%%%%%%%%%%%%%%%%%%%%%%%%%%%
\noindent
{\bf Basic}
\vskip 0.4cm

\noindent
Use  inverse quantile regression to estimate IV median regression of {\tt y} on
exogenous {\tt x1} and endogenous {\tt d1} with instruments {\tt z1} and {\tt
z2}

\vskip 0.2cm
{\tt \ivqregiqr\ y x1 (d1 = z1 z2)}

\vskip 0.2cm 
\noindent
As above, but estimate the 0.75 quantile

\vskip 0.2cm 
{\tt \ivqregiqr\ y x1 (d1 = z1 z2), quantile(0.75)}

\vskip 0.2cm 
\noindent
As above, but estimate the 0.1, 0.2, ..., 0.9 quantiles

\vskip 0.2cm 
{\tt \ivqregiqr\ y x1 (d1 = z1 z2), quantile(10(10)90)}

\vskip 0.2cm
\noindent
Use the smoothed estimating equation estimator to estimate the 0.6 quantile
regression of {\tt y} on exogenous {\tt x1} and endogenous {\tt d1} and {\tt d2}
with instruments {\tt z1} and {\tt z2}

\vskip 0.2cm 
{\tt \ivqregsee\ y x1 (d1 d2 = z1 z2), quantile(0.6)}

\vskip 0.2cm
\noindent
As above, but estimate the 0.1, 0.2, ..., 0.9 quantiles

\vskip 0.2cm 
{\tt \ivqregsee\ y x1 (d1 d2 = z1 z2), quantile(10(10)90)}

\vskip 0.5cm

%%%%%%%%%%%%%%%%%%%%%%%%%%%%%%%%%%
\noindent
{\bf Advanced options for inverse quantile regression estimator}
\vskip 0.2cm

\vskip 0.2cm
\noindent
Use 50 grid points in the inverse quantile regression to estimate the IV 0.5 and
0.75 quantile regression

\vskip 0.2cm 
{\tt \ivqregiqr\ y x1 (d1 = z1 z2), ngrid(50) quantile(50 75)}

\vskip 0.2cm
\noindent
As above, but construct grid points between 1 and 5 for all the quantiles

\vskip 0.2cm 
{\tt \ivqregiqr\ y x1 (d1 = z1 z2), ngrid(50) quantile(50 75) bound(1 5)}

\vskip 0.2cm
\noindent
As above, but construct grid using different bounds for different quantiles

\vskip 0.2cm 
{\tt \ivqregiqr\ y x1 (d1 = z1 z2), ngrid(50) quantile(50 75)}  
\hskip 2cm
\textbackslash\textbackslash\textbackslash

\hskip 3cm  {\tt bound(1 5, at(50)) bound(2 6, at(75))}

%%%%%%%%%%%%%%%%%%%%%%%%%%%%%%%%%%
\vskip 0.5cm
\noindent
{\bf Advanced options for smoothed estimating equation estimator}

\vskip 0.2cm
\noindent
Use $2$ as the bandwidth in the smoothed estimating equation estimator to
estimate the IV 0.5 and 0.75 quantile regression

\vskip 0.2cm 
{\tt \ivqregsee\ y x1 (d1 d2 = z1 z2), quantile(50 75) bwidth(2)}


\vskip 0.2cm
\noindent
As above, but use different bandwidths for different quantiles

\vskip 0.2cm 
{\tt \ivqregsee\ y x1 (d1 d2 = z1 z2), quantile(50 75)}	
\hskip 2cm
\textbackslash\textbackslash\textbackslash

\hskip 3cm {\tt bwidth(2, at(50)) bwidth(1, at(75)) }


\clearpage
%------------------------------------------------------------------------------
\section{Post-estimation of {\ivqreg}} \label{sec:post_syntax}
%------------------------------------------------------------------------------
\subsection{Overview}

The following postestimation commands are of particular interest after {\ivqreg}.

\vskip 0.5cm
\begin{tabular}{ll}
\hline
Commands & Description \\
\hline
{\tt estat coefplot} & plot coefficients and their confidence intervals at
different quantiles\\
{\tt estat \underline{endogef}fects} & process test of no effect, constant effect, stochastic
dominance, and endogeneity \\
*{\tt estat dualci} & provide the dual-confidence interval for the endogenous 
variables \\
*{\tt estat waldplot} & plot Wald statistics corresponding to each grid point
\\
\hline
\end{tabular}

\vskip 0.5cm
Note :
\begin{itemize}
  \item {\tt estat waldplot} and {\tt estat dualci} are allowed only after {\tt
    ivqregress iqr}.
\end{itemize}

\vskip 0.5cm

\noindent
The following postestimation commands are also available after {\tt ivqreg}

\begin{tabular}{ll}
\hline
Commands & Description \\
\hline
{\tt contrast}         &contrasts and ANOVA-style joint tests of estimates \\
{\tt estat summarize}  &summary statistics for the estimation sample \\
{\tt estat vce}        &variance-covariance matrix of the estimators (VCE) \\
{\tt estimates}        &cataloging estimation results \\
{\tt forecast}         &dynamic forecasts and simulations \\
{\tt lincom}           &point estimates, standard errors, testing, and inference
			for linear \\
		 &  combinations of coefficients \\
{\tt margins}          &marginal means, predictive margins, marginal effects,
			and average marginal \\ 
		  &  effects \\
{\tt marginsplot}      &graph the results from margins (profile plots,
		interaction plots, etc.) \\
{\tt nlcom}            &point estimates, standard errors, testing, and inference
			for nonlinear \\ &  combinations of coefficients \\
{\tt predict}          &predictions and their SEs, residuals, etc. \\
{\tt predictnl}        &point estimates, standard errors, testing, and inference
			for generalized \\ 
			&  predictions \\
{\tt pwcompare}        &pairwise comparisons of estimates \\
{\tt test}             &Wald tests of simple and composite linear hypotheses \\
{\tt testnl}           &Wald tests of nonlinear hypotheses \\
\hline
\end{tabular}


\clearpage
%------------------------------------------------------------------------------
\subsection{Syntax of {\tt estat coefplot}}	
%------------------------------------------------------------------------------
{\tt estat coefplot} plots the estimated coefficients and their confidence
intervals after {\tt ivqreg}. 

\vskip 0.5cm
\noindent
The syntax is 
\vskip 0.5cm
{\tt estat coefplot} [{\it varname}] [, {\it options}]

\vskip 0.5cm
\noindent
Notes: 

\begin{enumerate}
\item {\it varname} specifies the coefficient for the variable in the model.
By default, it plots the coefficients for the first endogenous variables
specified in the model.

\item The options are

\begin{tabular}{ll}
\hline
Options & Description \\
\hline
{\bf Main} \\
\quad {\tt noci} & do not plot the confidence intervals \\
\quad {\tt no2sls} & do not plot the 2SLS estimates \\
\\
{\bf Scatter plot} \\
\quad {\it marker\_options} & change look of markers (color, size, etc.) \\
\quad {\it connect\_options}& change look of lines or connecting method\\
\\ 
{\bf CI plot}
\\
\quad {\tt ciopts({\it area\_options})} & affect rendition of the pointwise
confidence interval\\
\\
{\bf Reference line}
\\
\quad {\tt lineopts({\it cline\_options})} & affect rendition of reference line
identifying the 2SLS estimates\\
\\
{\bf Others} \\
\quad {\it twoway\_options} & titles, legends, axes, added lines  and text,
regions, \\ & name, aspect ratio, etc. \\
\hline
\end{tabular}

\end{enumerate}

\noindent
where {\it marker\_options} and {\it connect\_options} are defined in {\tt
scatter}; 
{\it area\_options} and {\it twoway\_options} are defined {\tt rarea} and {\tt
twoway}, respectively.


\clearpage
%------------------------------------------------------------------------------
\subsection{Syntax of {\tt estat endogeffects}}
%------------------------------------------------------------------------------
{\tt estat endogeffects} tests the quantile process with different hypotheses 
for the coefficients on the endogenous treatment variables. In particular,
{\tt estat endogeffects} provides a test for four different hypotheses.  The null
hypotheses are 
\begin{enumerate}
  \item The hypothesis of no effect: the treatment has no effect on the
    outcome.

  \item The hypothesis of constant effect: the treatment effect does not vary at
    different quantiles.

  \item The dominance hypothesis: the effects are unambiguously beneficial.

  \item The exogeneity hypothesis: the treatment is exogenous.
\end{enumerate}

\vskip 0.5cm
\noindent
The syntax is 

\vskip 0.5cm
{\tt estat endogeffects} [{\it varlist}] [, {\it options}]

\vskip 0.5cm
\noindent
where {\it varlist} is the endogenous variables. By default, {\it varlist}
refers to all the endogenous variables specified in {\tt ivqregress} when
fitting the model.

\vskip 0.5cm
\noindent
The options are

\vskip 0.5cm
\begin{tabular}{ll}
\hline
Options & Description \\
\hline
{\tt \underline{l}evel(\#)} & confidence level of a test; default is $0.95$ \\
{\tt reps(\#)} & perform \# bootstrap replications; default is reps(100) \\
{\tt rseed(\#)}&  set random-number seed to \# \\
{\tt all} & test four hypotheses; the default \\
{\tt \underline{noef}fect} & test of no effect \\
{\tt \underline{cons}tant} & test of constant effect \\
{\tt \underline{dom}inance} & test of stochastic dominance \\
{\tt \underline{exog}eneity} & test of exogeneity \\
\hline
\end{tabular}

\vskip 0.5cm
\noindent


\clearpage
%------------------------------------------------------------------------------
\subsection{Syntax of {\tt estat waldplot}}
%------------------------------------------------------------------------------
{\tt estat waldplot} plots the Wald statistic corresponding to each grid point
after {\tt ivqregress iqr}. 

\vskip 0.5cm
\noindent
The syntax is 
\vskip 0.5cm
{\tt estat waldplot} [, {\it options}]

\vskip 0.5cm
\noindent
Notes: 

\begin{enumerate}

\item The options are the following:

\begin{tabular}{ll}
\hline
Options & Description \\
\hline
{\bf Main} \\
\quad {\tt \underline{q}uantile(\#)} & plot Wald statistics for the \#-th
quantile estimation \\
\quad {\tt \underline{l}evel(\#)} & set confidence level; default is {\tt
level(95)} \\
\\
{\bf Scatter plot} \\
\quad {\it marker\_options} & change look of markers (color, size, etc.) \\
\quad {\it connect\_options}& change look of lines or connecting method\\
\\
{\bf Dual CI plot}
\\
\quad {\tt ciopts({\it area\_options})} & affect rendition of the dual 
confidence interval plot\\
\\
{\bf Others} \\
\quad {\it twoway\_options} & titles, legends, axes, added lines  and text,
regions, \\ & name, aspect ratio, etc. \\
\hline
\end{tabular}

\noindent
where {\it marker\_options} and {\it connect\_options} are defined in {\tt
scatter}; 
{\it area\_options} and {\it twoway\_options} are defined {\tt rarea} and {\tt
twoway}, respectively.

\end{enumerate}

\noindent


%------------------------------------------------------------------------------
\subsection{Syntax of {\tt estat dualci}}	
%------------------------------------------------------------------------------
{\tt estat dualci} computes the dual confidence interval for the endogenous
variables after {\tt ivqregress iqr}. {\tt estat dualci}
implements the dual confidence interval method proposed in
\cite{Chernozhukov2008}. The dual confidence interval is robust to the weak
instruments, and it is usually wider than the traditional CI.

\vskip 0.5cm
The syntax is 
\vskip 0.5cm
{\tt estat dualci} [, {\tt \underline{l}evel(\#)} {\it display\_options}]
\vskip 0.5cm

\noindent
{\it display\_options}: 
  {\tt noci}, 
  {\tt nopvalues}, 
  {\tt noomitted}, 
  {\tt vsquish}, 
  {\tt noemptycells}, 
  {\tt baselevels}, 
  {\tt allbaselevels},
  {\tt nofvlabel}, 
  {\tt fvwrap(\#)}, 
  {\tt fvwrapon({\it style})}, 
  {\tt cformat({\it \%fmt})}, 
  {\tt pformat({\it \%fmt})},
  {\tt sformat({\it \%fmt})}, 
  and {\tt nolstretch}; see [R] Estimation options.

%------------------------------------------------------------------------------
\subsection{Syntax of {\tt predict}}
%------------------------------------------------------------------------------

The syntax for {\tt predict} is
\vskip 0.5cm
{\tt predict} [{\it type}] {\it newvar} [{\it if}] [{\it in}] 	
[, {\tt \underline{eq}uation({\it eqno})} {\it statistic}]
\vskip 0.5cm

\begin{tabular}{ll}
\hline
statistic& Description \\
\hline 
{\bf Main} \\
\quad {\tt xb}       &linear prediction; the default \\
%\quad {\tt stdp}      &standard error of the linear prediction \\
%\quad {\tt stddp}     &standard error of the difference in linear predictions \\
\quad {\tt \underline{res}iduals} &residuals \\
\hline
\end{tabular}

\vskip 0.5cm
\noindent
These statistics are available both in and out of sample; type {\tt predict ...
if e(sample) ...} if wanted only for the estimation sample.

The options are
\begin{itemize}

\item {\tt xb}, the default, calculates the linear prediction.

%\item {\tt stdp} calculates the standard error of the linear prediction.

%\item {\tt stddp} is allowed only after you have fit a model at least two
%quantiles.  The standard error of the difference in linear predictions between
%equations 1 and 2 is calculated.

\item {\tt residuals} calculates the residuals, that is, y - xb.

\item {\tt equation({\it eqno})} specifies the equation to which
you are making the calculation.

	{\tt equation()} is filled in with one {\it eqno} for the {\tt xb} and
	{\tt residuals} options.  {\tt equation(\#1)} would mean that the
	calculation is to be made for the first equation, {\tt equation(\#2)}
	would mean the second, and so on.  You could also refer to the equations
	by their names.  {\tt equation(q10)} would refer to the equation named
	{\tt q10} and {\tt equation(q50)} to the equation named {\tt q50}.

	If you do not specify {\tt equation()}, results are the same as if you
	had specified {\tt equation(\#1)}.

%	To use {\tt stddp}, you must specify two equations.  You might specify
%	{\tt equation(\#1, \#2)} or {\tt equation(q80, q20)} to indicate the
%	80th and 20th quantiles.

\end{itemize}



%------------------------------------------------------------------------------
\subsection{Syntax of {\tt margins}}
%------------------------------------------------------------------------------

Syntax for {\tt margins}

\vskip 0.5cm

{\tt margins} [{\it marginlist}] [, {\it options}]

\vskip 0.5cm

{\tt margins} [{\it marginlist}] , {\tt predict({\it statistic} ...)} [{\it
options}]

\vskip 0.5cm

\begin{tabular}{ll}
\hline
statistic&          Description \\
\hline
{\tt xb}       & linear prediction; the default \\
%{\tt stdp}     & not allowed with margins \\
%{\tt stddp}    & not allowed with margins \\
{\tt residuals}& not allowed with margins \\
\hline
\end{tabular}

\vskip 0.5cm
\noindent
Statistics not allowed with margins are functions of stochastic quantities other
than {\tt e(b)}.  For the full syntax, see [R] {\tt margins}.
